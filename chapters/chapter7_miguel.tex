\documentclass[../main.tex]{subfiles}
\begin{document}
\section{Recodatorio teoría de cuerpos}
Sea $K \subset L$ una extensión de cuerpos. Un elemento $u \in L$ es \emph{transcendente} sobre $K$ si ningún polinomio no nulo de $K[T]$ se anula sobre este, es decir, $p(u)=0 \Rightarrow p=0 \in K[T]$.

Dados $\left\{u_{1}, \ldots, u_{s}\right\} \subset L$ se dice que son \emph{algebraicamente independientes} sobre $K$ si no hay ningún polinomio en $p\in K[T_1,\dots, T_s]$ no nulo que tal que $p\left(u_{1}, \ldots, u_{s}\right)=0$.
Se desprende de la definición que si $\{u_{1}, \ldots, u_{s}\}$ son algebraicamente independientes entonces el menor subcuerpo de $L$ que contiene a $K$ y a los $u_i$, que notaremos $K(u_{1}, \ldots, u_{s})$, es isomorfo a un anillo de polinomios:
$K(u_{1}, \ldots, u_{s}) \cong K[T_{1}, \ldots, T_{s}]$.

Sea $K \subset L=K\left(\theta_{1}, \ldots, \theta_{n}\right)$ una extensión de cuerpos finitamente generada, entonces existe $\{\theta_{i_{1}}, \ldots, \theta_{i_{r}}\}\subset \{\theta_{1}, \ldots, \theta_{n}\}$ algebraicamente
independientes sobre $K$ tales que $K\left(\theta_{i_{1}}, \ldots, \theta_{i_{r}}\right) = L$. El conjunto $\left\{\theta_{i_{1}}, \ldots, \theta_{i_{r}}\right\}$ se denomina \emph{base de trascendencia}. Su cardinal es común a todas las bases de trascendencia y se denomina \emph{grado de trascendencia} de $L$ sobre $K$.

Si $A \subset B$ son D.I se puede hablar de grado de trascendencia de $B$ sobre $A$ como el correspondiente a la extensión entre sus respectivos cuerpos de fracciones.

\begin{remark}[\textbf{(Notación)}]
Durante el capítulo, las letras mayúsculas denotarán elementos algebraicamente independientes.
\end{remark}

\section{Lema de normalización de Noether}
\begin{lemma}
Sea $K$ un cuerpo infinito y $F\in K[X_1, \dots, X_n]$ un polinomio homogéneo no nulo. Entonces existe $(a_1, \dots, a_{n-1}) \in K^{n-1}$ tales que $F(a_1, \dots, a_{n-1},1)\neq 0$.
\end{lemma}
\begin{proof}
Por inducción sobre $n$. El caso para $n=1$ es trivial porque $F(X_1) = aX_1$ con $a\neq 0$. Suponemos el resultado cierto para $n>1$ y escribimos $F = \sum_{k=0}^d F_k X_1^k$ donde $F_k \in K[X_2,\dots, X_n]$ son polinomios homogéneos de grado $d-k$.  Si $F\neq 0$, al menos algún $F_k$ es no nulo.
Por inducción podemos escoger $(a_2, \dots, a_{n-1}) \in K^{n-2}$ tal que $F_k(a_2, \dots, a_{n-1},1)\neq 0$. Entonces $F(.,a_2, \dots, a_{n-1},1) \in K[X_1]$ es un polinomio no nulo y tiene por tanto un número finito de raíces. Como $K$ es infinito, existe algún $a_1 \in K$ tal que $F(a_1,a_2, \dots, a_{n-1},1) \neq 0$.
\end{proof}
\begin{remark}\label{obs_import}
Sea $K\left[X_{1}, \ldots, X_{n}\right]$ el anillo de polinomios sobre un cuerpo $K$ infinito,  y sea $F \in K\left[X_{1}, \ldots, X_{n}\right] \setminus K$ un polinomio no constante. Podemos escribir $F$ en componentes homogéneas: $F=F_{r}+\ldots+F_{s}$ donde el subíndice indica el grado de la componente, con $r \leq s$ y con $F_{s} \neq 0 .$

Por el lema, existe $\left(\alpha_{1}, \ldots, \alpha_{n-1}\right) \in K^{n-1}$ tal que $F_{s}\left(\alpha_{1}, \ldots, \alpha_{n-1}, 1\right) \neq 0$. Realizamos el siguiente cambio de variable
$$
\begin{cases}
  Y_{i}:=X_{i}-\alpha_{i} X_{n} & i=1 \dots n-1\\
  Y_n := X_n
\end{cases}
$$
Entonces, podemos sustituir el cambio en $F$ y separar en componentes homogéneas:
\begin{multline}
  F\left(Y_{1}+\alpha_{1} X_{n}, \ldots, Y_{n-1}+\alpha_{n-1} X_{n}, X_{n}\right)=\\
  =\left (\sum_{i_{1}+\cdots+i_{n}=s} a_{i_{1} i_{2} \ldots i_{n}}\left(Y_{1}+\alpha_{1} X_{n}\right)^{i_{1}} \left(Y_{1}+\alpha_{1} X_{n}\right)^{i_{2}} \ldots \left(Y_{n-1}+\alpha_{n-1} X_{n}\right)^{i_{n-1}}  X_{n}^{i_{n}} \right )+ G = \\
  =F_{s}\left(\alpha_{1}, \ldots, \alpha_{n-1}, 1\right) X_{n}^{s}+ G = cX_n^s+G
\end{multline}
donde $G$ agrupa los términos de menor grado en $X_n$. De otra forma, $c X_{n}^{s}+ G-F=0$ y acabamos de comprobar que $A:=K\left[Y_{1}, \ldots, Y_{n-1}, F\right] \subset K\left[X_{1}, \ldots, X_{n}\right]$ es una extensión entera, ya que $X_{n}$ es entero sobre $A$ y $\left\{Y_{1}, \ldots, Y_{n-1}, F\right\}$ son algebraicamente independientes sobre $K$.
\end{remark}

\begin{theorem}[\textbf{(de normalización)}]
Sea $K$ cuerpo infinito, $A=K\left[x_{1}, \ldots, x_{n}\right]$ una $K$-álgebra finitamente generada\footnote{Es útil pensarlo como un cociente del anillo de polinomios por un ideal.} e $I$ un ideal propio de $A$. Entonces existen $d \leq n, \delta \leq d, Y_{1}, \ldots, Y_{d} \in A$ algebraicamente independientes
sobre $K$ tal que
\begin{enumerate}
  \item la extensión $A \supset A^{\prime}:=K\left[Y_{1}, \ldots, Y_{d}\right]$ es entera; ie. A es un $K\left[Y_{1}, \ldots, Y_{d}\right]$ -módulo finito.
  \item $I \cap K[Y_{1}, \ldots, Y_{d}]=\left\langle Y_{\delta+1}, \ldots, Y_{d}\right\rangle$,
  \item los $Y_{1}, \ldots, Y_{\delta}$ son combinaciones lineales de $\operatorname{los} x_{1}, \ldots, x_{n}$ con coeficientes en $K$.
\end{enumerate}
\end{theorem}

\begin{remark}
Si $\delta=d$, significa $I=\langle 0\rangle$. Además, $d=0$ cuando $A$ es una extensión algebraica de $K$.
\end{remark}

\begin{remark}
Cuando $I=\langle 0\rangle$,se predican (i) y (ii) se tiene el \emph{Lema de normalización de Noether} usual.
\end{remark}

\begin{proof}
Distinguiremos varios casos:

\paragraph{Caso 1:} Supongamos que los generadores de $A$ son algebraicamente independientes, ie. $A\cong K[X_1, \dots, X_n]$, y que el ideal es principal $I = \langle f \rangle$. Podemos hacer el cambio de variable de la observación \ref{obs_import} y así sabemos que $A':= K[Y_1, \dots, Y_{n-1}, Y_n] \subset K[X_1, \dots, X_n]$ es una extensión entera, donde hemos llamado $Y_n := f$. Como el grado de trascendencia de $K[X_1, \dots, X_n]$ sobre $K$ es $n$ por ser algebraicamente independientes, tenemos que $Y_1, \dots, Y_n$ son algebraicamente independientes.

Entonces $A'$ es un anillo de polinomios con coeficientes en un cuerpo, luego es un $DFU$ y por tanto es integramente cerrado sobre su cuerpo de fracciones.

Comprobamos que $A' \cap I = \langle Y_n \rangle_{A'}$. Recordamos que $Y_n = f$, luego el contenido $\supset$ es automático. Sea ahora $h = g f \in I \cap A'$ (es de esa forma por pertenecer a $I$). Entonces $g$ pertence a $K_{A'}$ el cuerpo de fracciones de $A'$ y es entero sobre $A'$, por lo tanto $g \in A'$ así efectivamente $h$ es un múltiplo de $Y_n$ en $A'$.

\paragraph{Caso 2:} Supongamos de nuevo que $A= K[X_1,\dots, X_n]$ es un anillo de polinomios y ahora tomemos un ideal $I$ arbitrario. Por inducción sobre $n$. Si $n=1$, $K[X_1]$ es DIP y estamos en caso 1. Supongamos el resultado cierto para todo $k< n$ y lo probamos para $n$.

Si $I=\langle 0\rangle$ es trivial. En otro caso, sea $F \in I \backslash\{0\}$, y realizando la construcción anterior obtenemos la extensión entera $B:=K\left[Y_{1}, \ldots, Y_{n-1}, Y_{n}=F\right] \subset A$ que cumple que $I \cap B \supset\langle Y_{n}\rangle_{B}$.

Sea $A_{1}:=K[Y_{1}, \ldots, Y_{n-1}]$ y el ideal
$I_{1}:=I \cap A_{1}$. Por la hipótesis de inducción existen $d'$ y $\delta'$ con $\delta' \leq d'$ tales que
\begin{itemize}[]
  \item[(a)] $K\left[T_{1}, \ldots, T_{d^{\prime}}\right] \subset A_{1}$ es una extensión entera,
  \item[(b)] $I_{1} \cap K\left[T_{1}, \ldots, T_{d^{\prime}}\right]=\left\langle T_{\delta^{\prime}+1}, \ldots, T_{d}^{\prime}\right\rangle, \mathrm{y}$,
  \item[(c)] $T_{1}, \ldots, T_{\delta^{\prime}}$ son
  combinaciónes lineales de $Y_{1}, \ldots, Y_{n-1}$
\end{itemize}

Por (a) tenemos $d^{\prime}=n-1$ y
$$A^{\prime}:=K\left[T_{1}, \ldots, T_{n-1}, Y_{n}\right] \subset K\left[Y_{1}, \ldots, Y_{n-1}, Y_{n}\right] \subset A$$
donde $A$ es $A^{\prime}$ módulo finito, por transitividad.

Tomemos $d:=n$, $\delta:=\delta^{\prime}$, y $T_{n}:=Y_{n}$. Es claro que $T_{\delta^{\prime}+1}, \ldots, T_{n-1}, Y_{n} \in I \cap A^{\prime} $. Asimismo si
$g \in I \cap A^{\prime}$, entonces $$g=g_{1}\left(T_{1}, \ldots, T_{n-1}\right)+Y_{n} g_{2}\left(T_{1}, \ldots, T_{n-1}, Y_{n}\right)$$
Como $Y_{n} \cdot g_{2} \in I \cap A^{\prime}$, se tiene que $$g_{1} \in I \cap K\left[T_{1}, \ldots, T_{n-1}\right]=I_{1} \cap K\left[T_{1}, \ldots, T_{n-1}\right]=\left\langle T_{\delta^{\prime}+1}, \ldots, T_{d^{\prime}}\right\rangle$$
Así $\left\langle T_{\delta^{\prime}+1}, \ldots, T_{n-1}, Y_{n}\right\rangle_{A^{\prime}}=I \cap A^{\prime}$.

Finalmente, $T_1, \dots, T_{\delta'}$ son combinaciones lineales de los $Y_1, \dots, Y_{n-1}$ que a su vez era combinaciones lineales de $X_1, \dots, X_n$, y tenemos el resultado.

\paragraph{Caso general:} Sea $A=K\left[x_{1} \ldots, x_{n}\right]$ e $/$ un ideal. Por caracterización general de álgebras finitamente generadas, existen ideales $J, I_0 \subset K[X_1, \dots, X_n]$ con $J \subset I_0$ tales que  $A \cong K\left[X_{1}, \ldots, X_{n}\right] / J$ e  $I=I_{0} / J$.

Aplicamos el caso 2 a $K\left[X_{1}, \ldots, X_{n}\right]$ y $J$. Se obtienen así $d'=n$, $\delta' \leq n$ y $Y_1, \dots Y_n \in A$ tales que
\begin{itemize}
  \item $K\left[Y_{1}, \ldots, Y_{n}\right] \subset K\left[X_{1}, \ldots, X_{n}\right]$ extensión entera,
  \item $J \cap K\left[Y_{1}, \ldots, Y_{n}\right]=\left\langle Y_{\delta^{\prime}+1}, \ldots, Y_{n}\right\rangle$
  \item $Y_{1}, \ldots, Y_{\delta^{\prime}}$ son combinaciones lineales de los $X_{i}$.
\end{itemize}

Así $K\left[Y_{1}, \ldots, Y_{\delta^{\prime}}\right] \cap J=\langle 0\rangle$ y $K\left[Y_{1}, \ldots, Y_{\delta^{\prime}}\right] \hookrightarrow A=K\left[X_{1}, \ldots, X_{n}\right] / J$ es una extensión entera.

Sea $d:=\delta'$. Aplicando el caso 2 a $A_{1}:=K\left[Y_{1}, \ldots, Y_{d}\right]$ y a $I_{1}:=I_{0} \cap K[Y_{1}, \ldots, Y_{d}]$ existen $\delta'', d''$ con $\delta'' \leq d''$, y existen $T_1, \dots, T_{d''} \in A_{1}$ de manera que
\begin{itemize}
  \item $K\left[T_{1}, \ldots, T_{d^{\prime \prime}}\right] \subset A_{1}$ es extensión entera,
  \item $I_{1} \cap K\left[T_{1}, \ldots, T_{d^{\prime \prime}}\right]=\left\langle T_{\delta^{\prime \prime}+1}, \ldots, T_{d^{\prime \prime}}\right\rangle$\footnote{Observamos que $d^{\prime \prime}=d$.},
  \item $T_{1}, \ldots, T_{\delta^{\prime \prime}}$ son combinaciones lineales de $Y_{1}, \ldots, Y_{d}$
\end{itemize}

Con todo esto, finalmente tenemos que
\begin{enumerate}
  \item  $K\left[T_{1}, \ldots, T_{d}\right] \subset A_{1} \hookrightarrow A$ son extensiones enteras,
  \item  $I \cap K\left[T_{1}, \ldots, T_{d}\right]=\left(I_{0} \cap K\left[Y_{1}, \ldots, Y_{d}\right]\right) \cap K\left[T_{1}, \ldots, T_{d}\right]=\left\langle T_{\delta^{\prime \prime}+1}, \ldots, T_{d}\right\rangle$,
  \item las $T_1, \dots, T_{\delta''}$ son combinaciones lineales de las $X_1, \dots, X_n$.
\end{enumerate}
como queríamos demostrar.
\end{proof}

\end{document}
